\documentclass[12pt,letterpaper]{article}
\usepackage{amsfonts}
\usepackage{amssymb}
\usepackage{graphicx}
\usepackage[margin=3cm]{geometry}
\usepackage{fancyhdr}
\usepackage{mathtools}
\usepackage{verbatim}
\begin{document}

\pagestyle{fancy}
\lhead{Tyler Ayrton Stank}
\chead{Assignment}
\rhead{\today}

\begin{enumerate}

    %1
    \item
    A little playing around shows $c_0 = 1, c_1 = 0, c_2 = 1$.  The LFSR bits are then as follows:
    \begin{verbatim}
        010
        100
        001
        011
        111
        110
    \end{verbatim}

    %2
    \item
    An LFSR of length $i$ produces at most $2^i-1$ output bits before restarting a cycle and the period of this LFSR's output is 21, so we need an LFSR function of length 5.
    Thus, we can set up our system of equations as follows:
    \begin{comment}
    \[
    \begin{cases}
        x_6 = c_1 x_5 + c_2 x_4 + c_3 x_3 + c_4 x_2 + c_5 x_1\\
        x_7 = c_1 x_5 + c_2 x_4 + c_3 x_3 + c_4 x_2 + c_5 x_1\\
        x_8 = c_1 x_5 + c_2 x_4 + c_3 x_3 + c_4 x_2 + c_5 x_1\\
        x_9 = c_1 x_5 + c_2 x_4 + c_3 x_3 + c_4 x_2 + c_5 x_1\\
        x_{10} = c_1 x_5 + c_2 x_4 + c_3 x_3 + c_4 x_2 + c_5 x_1\\
        x_{11} = c_1 x_5 + c_2 x_4 + c_3 x_3 + c_4 x_2 + c_5 x_1\\
    \end{cases}
    \]
    \end{comment}
    \[
    \left\{
        \setlength\arraycolsep{2pt}
        \begin{array}{ l @{{}={}} l c l c l c l c l }
        x_6    & c_1 x_5    &+& c_2 x_4 &+& c_3 x_3 &+& c_4 x_2 &+& c_5 x_1\\
        x_7    & c_1 x_6    &+& c_2 x_5 &+& c_3 x_4 &+& c_4 x_3 &+& c_5 x_2\\
        x_8    & c_1 x_7    &+& c_2 x_6 &+& c_3 x_5 &+& c_4 x_4 &+& c_5 x_3\\
        x_9    & c_1 x_8    &+& c_2 x_7 &+& c_3 x_6 &+& c_4 x_5 &+& c_5 x_4\\
        x_{10} & c_1 x_9    &+& c_2 x_8 &+& c_3 x_7 &+& c_4 x_6 &+& c_5 x_5\\
        \end{array}
        \right.
    \]
    Substituting values from the beginning of the LFSR ($x_1 = 0, x_2 = 0, x_3=1, ...$), we get the following easily solvable system:
    \[
    \left\{
        \setlength\arraycolsep{2pt}
        \begin{array}{ l @{{}={}} lclclclcl }
        0      & c_1  0     &+& c_2  0  &+& c_3  1  &+& c_4  0  &+& c_5  0 \\
        0      & c_1  0     &+& c_2  0  &+& c_3  0  &+& c_4  1  &+& c_5  0 \\
        1      & c_1  0     &+& c_2  0  &+& c_3  0  &+& c_4  0  &+& c_5  1 \\
        1      & c_1  1     &+& c_2  0  &+& c_3  0  &+& c_4  0  &+& c_5  0 \\
        1      & c_1  1     &+& c_2  1  &+& c_3  0  &+& c_4  0  &+& c_5  0 \\
        \end{array}
        \right.
    \]

    The first four equations each tell us the value of one variable: $c_3 = c_4 = 0, c_1 = c_5 = 1$, so we only need to solve for $c_2$.  This can be done easily enough by substituting the values for the other four variables into the fifth equation.  This yields
    $$1 + c_2 + 0 + 0 + 0 = 1$$
    so we can see easily that $c_2 = 0$.
    Our solution then is
    \[
    \left\{
        \begin{array}{ l @{{}={}} l }
        c_1 & 1\\
        c_2 & 0\\
        c_3 & 0\\
        c_4 & 0\\
        c_5 & 1\\
        \end{array}
        \right.
    \]

    %3
    \item

    %4
    \item


\end{enumerate}

\end{document}
