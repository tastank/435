\documentclass[12pt,letterpaper]{article}
\usepackage{amsfonts}
\usepackage{amssymb}
\usepackage{graphicx}
\usepackage[margin=3cm]{geometry}
\usepackage{fancyhdr}
\usepackage{mathtools}
\begin{document}

\pagestyle{fancy}
\lhead{Tyler Ayrton Stank}
\chead{Assignment}
\rhead{\today}

\begin{enumerate}

    %1
    \item
        For this problem, a shorthand notation is used in which ciphertext is capitalized (e.g. RSZWO) and plaintext is lowercase (e.g. <plaintextexample>).\\\\
        A frequency analysis program written in C tells us the frequency distribution is as follows:\\
            B -- 2\\
            C -- 18\\
            D -- 7\\ 
            G -- 13\\
            H -- 11\\
            J -- 3\\
            K -- 6\\
            L -- 1\\
            N -- 13\\
            O -- 18\\
            P -- 14\\
            Q -- 2\\
            R -- 11\\
            S -- 16\\
            T -- 3\\
            V -- 7\\
            W -- 11\\
            X -- 5\\
            Y -- 4\\
            Z -- 19\\
        Six letters were not used: A, E, F, I, M, U.
        We thus replace Z with e, as Z is most common in the ciphertext and e most common in English plaintext.
        A little intuition tells us `the' is a common word with which to start a phrase, and more observation that `th' and `he' are the most common bigrams in English - which are both supported by replacing R with t and S with h (as RS and SZ are both common in the ciphertext).
        Reusing this concept, we find W occurs after e 7 of e's 19 occurrences - leading us to believe it must be r.
        


    %2
    \item

\end{enumerate}

\end{document}
