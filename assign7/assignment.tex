\documentclass[12pt,letterpaper]{article}
\usepackage{amsfonts}
\usepackage{amssymb}
\usepackage{graphicx}
\usepackage[margin=3cm]{geometry}
\usepackage{fancyhdr}
\usepackage{mathtools}
\usepackage{verbatim}
\begin{document}

\pagestyle{fancy}
\lhead{Tyler Ayrton Stank}
\chead{Assignment}
\rhead{\today}

\begin{enumerate}

    %1
    \item
        In the RSA cryptosystem, $de\equiv 1 \pmod {(p-1)(q-1)}$.~Thus, given $p$, $q$, and $d$ or $e$, we can easily find the missing value.~
        First, we calculate $(p-1)(q-1) = (19-1)(29-1) = 18\times28 = 504$.~From here, we must simply calculate the modular inverse of $e = 17$ to find $d$.~
        Because this is a lengthy process and we've been over it in class before (using the Extended Euclidean Algorithm), I won't elaborate on how it is found, but this inverse is 89.

    %2
    \item
        \begin{enumerate}
            %2a
            %knowing nothing about the messages themselves, I'm guessing this question is essentially ceil(log_2(132)) = 8, to give each message a different binary number
            \item
                To uniquely specify 132 elements, we need at least 8 binary digits; $2^7=128$, $2^8=256$.

        \end{enumerate}

    %3
    \item

    %4
    \item


\end{enumerate}

\end{document}
