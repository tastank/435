\documentclass[12pt,letterpaper]{article}
\usepackage{amsfonts}
\usepackage{amssymb}
\usepackage{graphicx}
\usepackage[margin=3cm]{geometry}
\usepackage{fancyhdr}
\usepackage{mathtools}
\usepackage{verbatim}
\begin{document}

\pagestyle{fancy}
\lhead{Tyler Ayrton Stank}
\chead{Assignment}
\rhead{\today}

\begin{enumerate}

    %1
    \item
        In the RSA cryptosystem, $de\equiv 1 \pmod {(p-1)(q-1)}$.~Thus, given $p$, $q$, and $d$ or $e$, we can easily find the missing value.~
        First, we calculate $(p-1)(q-1) = (19-1)(29-1) = 18\times28 = 504$.~From here, we must simply calculate the modular inverse of $e = 17$ to find $d$.~
        Because this is a lengthy process and we've been over it in class before (using the Extended Euclidean Algorithm), I won't elaborate on how it is found, but this inverse is 89.

    %2
    \item
        \begin{enumerate}
            %2a
            %knowing nothing about the messages themselves, I'm guessing this question is essentially ceil(log_2(132)) = 8, to give each message a different binary number
            \item
                To uniquely specify 132 elements, we need at least 8 binary digits; $2^7=128$, $2^8=256$.
            \item
            \item
                $143 = 11 \times 13$, 
                So $p$ and $q$ are 11 and 13.~
                This means
                $$(p-1)(q-1) = 10 * 12 = 120$$
                $$ed \equiv 1 \pmod {120}$$
                $$d = e^{-1} \pmod {120}$$
                $$d = 11$$
                This can be confirmed: $11 \times 11 = 121 \equiv 1 \pmod {120}$.
            \item
                $$c = M^e \pmod {120} = 5^{11} \pmod {120}$$
                $$ = (5^3)^3 \times 5^2 \pmod {120}$$
                $$ = 5^3 \times 5^2 \pmod {120}$$
                $$ = 5 \times 5^2 \pmod {120}$$
                $$ = 5$$
                And if we decrypt it, we know we will get 5 again, since in this special case the decryption function is the same as the encryption function.

        \end{enumerate}

    %3
    \item
        $K$, the common key calculated by Alex and Bob, is equal to $a^{xy}\pmod {p} = 7^{xy} \pmod {71}$.~
        Because 7 is a primitive root for $\mathbb{Z}^*_{71}$ we have $7^{70} \equiv 1 \pmod {71}$.
        So the question is now asking for two factorisations of 70.~
        Therefore, $(x,y) = {(7, 10), (2, 35)}$.

    %4
    \item
        \begin{enumerate}
            \item
                8 is the largest primitive root for $\mathbb{Z}^*_{11}$.~ %TODO: need to prove!
            \item
                $K$ is equal to $8^{xy} \pmod {11} = 8^{5 \times 3} \pmod {11} = 10$.
        \end{enumerate}


\end{enumerate}

\end{document}
